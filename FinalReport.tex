\documentclass{article}
\usepackage{setspace}   %Allows double spacing with the \doublespacing command
\usepackage{lipsum} % Add dummy text
\begin{document}

\title{Population prediction based on satellite imagery\\
Term Project Report} 

\date{December 2017}
\author{Yassine Kadiri, Santiago Novoa, Zsolt Pajor-Gyulai, Manuel Serrano}
\maketitle

\doublespacing


\section{Summary}
In this project, we attempted to build models predicting population density based on satellite imagery. We achieve this by comparing spatially disaggregated census data on the continental United States (that is excluding Alaska and U.S. territories) to satellite images of the particular region. We use several approaches: 
\begin{itemize}
\item[(1)] naive logistic regression on the vectorized satellite images; 
\item[(2)] convolutional neural network(CNN) built from scratch; \item[(3)] pre-trained CNN developed for image recognition (Vgg16).
\end{itemize}

While our models do not achieve particularly high accuracy, they show considerable lift corresponding to random guessing. Perhaps not surprisingly, the pre-trained neural network showed the best performance of achieving $62\%$ accuracy. [Finish this with final conclusions]

\section{Business understanding}
The 'business' in question in this case is mostly non-profit, governmental application. In order to allocate resources, government agencies require knowledge on the geographical distribution of the country's population. In the absence of direct measurements in years between two censuses, these entities are forced to rely on ad-hoc methods to model the evolution of the population since the last census year.

Our project explores the possibility of obtaining direct measurements of the population of a particular area by looking at Satellite imagery. Such capability would be of great value in many decision making processes such as urban development, evacuation planning, and gauging future demand for food, water, energy, and services. For example, according to the US General Accounting Office, more than 70 federal programs distribute tens of billions of dollars annually on the basis of population estimates.

Furthermore, censuses in many countries are non-representative due to limited civil registration systems or are outright fraudulent. In this case, having an independent way to estimate population could be beneficial in the optimal allocation of humanitarian aid or for clandestine purposes.

\section{Data}
\subsection{Data understanding}

\subsection{Data preparation}
\section{Modeling and evaluation}
Our goal in this section is to show that we can extract predictive value from our data, that is, our models provide a non-unit lift compared to random guessing.
\subsection{Logistic regression}
\subsection{CNN from scratch}
\subsection{Pretrained VGG architecture}
\subsection{Comparison ?}
\section{Deployment}
After a successful validation of our models, deployment should be in the form of an automated software implementation, where decision makers can obtain the figure on the population of a particular area by simply feeding the neural network the corresponding satellite images.

However, every governmental agency using our model for decision making should be aware of the imperfection of our predictions. Accordingly, whenever human life depends on the results (e.g. evacuation planning, or disaster relief funds), the policymakers should make sure to have appropriate cushion built into their actions. To provide further security, the cautious user should use our models in conjunction with other techniques and be alerted by huge discrepancies. This of course applies to all other applications with, perhaps, less severe consequences.

As features of areas populated by humans, in particular overall architecture, change rather slowly over time, one would expect a well trained model to be robust over a long period of time. Of course, the model should go through revalidation or, perhaps, retraining from time to time. A natural point for this to occur are the census years, when labeled data becomes available.

Unless it is the explicit goal of using the model (e.g. in clendestine applications), one has to be aware that obtaining satellite images of people's properties raises obvious privacy issues. The respectful user should use a low enough resolution such that the details of individual's properties cannot be extracted from the image.

\appendix
\section{Contribution of each team member to the project}
\subsection{Yassine Kadiri}
\subsection{Santiago Novoa}
\subsection{Zsolt Pajor-Gyulai}
\subsection{Manuel Serrano}

\end{document}